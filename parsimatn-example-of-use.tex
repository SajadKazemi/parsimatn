\documentclass[10pt,a5paper]{article}
\usepackage[a5paper,margin=1.5cm]{geometry}

\usepackage{amsmath}
\usepackage{amsfonts}
\usepackage{amssymb}
\usepackage{amsthm}


\theoremstyle{definition}
\newtheorem{definition}{تعریف}[section]

\theoremstyle{plain}
\newtheorem{proposition}[definition]{گزاره}

\usepackage[computeautoilg=on]{xepersian}
\settextfont[Numbers=Proportional]{Parsi Matn}
\settextdigitfont[Numbers=Proportional]{Parsi Matn}
\setmathdigitfont{Parsi Matn}

\SepMark{-}

\title{\textbf{فونت «پارسی متن» - \lr{``Parsi Matn''}}}
\author{ساده، کارا، آشنا، زیبا}

\date{%
گونۀ 1.2 آ
\quad - \quad \today \\
\lr{Version 2.1a \quad - \quad \latintoday}
}

\usepackage{ptext}
\begin{document}
\maketitle

\vspace*{1cm}


\begin{abstract}
ساخت و توسعۀ فونتِ رایگان، کاری است که انجام آن، با ترکیبِ «علاقه+بیکاری» شدنی است. مقداری از علاقۀ من به سیستم \TeX\ به این خاطر است که سیستمی مانند \TeX\ آن‌قدر از فونت‌ها شناخت دارد و در استفاده از قدرتِ فونت‌ها ورزیده است که حتی شناخته‌شده‌ترین نرم‌افزارهای حروف‌چینی و صفحه‌آرایی، هنوز برخی از امکاناتی که \TeX\ دهه‌ها پیش در استفاده از فونت‌ها به‌کار برده را به‌کار نمی‌برند.

آشنا، زیبا، ساده، کارا -- نام‌هایی که می‌توان از آن‌ها برای توصیف این فونت استفاده کرد. این فونت تلاشی بوده تا نوشتارهای پارسی که با \TeX\ و سیستم‌های برگرفته از آن حروفچینی می‌شوند، در کنار نمادها و فرمول‌های ریاضیاتی، ظاهری شایسته داشته‌باشند.
\end{abstract}

\newpage

\section{نمونه‌های گوناگون}
در این قسمت، متن‌هایی با محتواهای گوناگون نوشته‌شده، که به کاربر کمک می‌کند تا یک دیدِ کلی دربارۀ فونت داشته‌باشد.

\subsection{متن عادی}
با نگاه دقیق‌تر به قلم و کیفیت خوانایی آن و اثر آن بر روی مخاطب، می‌توانیم بپذیریم که تایپوگرافی و حروف‌چینی یک متن به رعایت اصول و ضوابط طراحی قلم بیشتر نیازمند است تا مسائل طراحی گرافیک. استفاده از قلم مناسب راحت خوانده شدن را به خواننده هدیه می‌کند و بی آن‌که خواننده حس کند که طراح در تنظیم حروف‌چینی هنرنمایی کرده است، محیط مناسبی برای خوانایی او ایجاد می‌شود. استفاده از قلم مناسب در یک متن به‌ضرورت نظر دارد و نه به هنر. ما از خواندن بسیاری از متن‌ها لذت می‌بریم، بی آن‌که شیفتهٔ شکل خاص طراحی قلم بشویم، یا حتی ظرایف طراحی آن در ذهن ما حک شود. اگرچه تنوع قلم در نرم‌افزارهای جدید فارسی رو به افزایش است اما لازم به توجه است که در هر یک از این نرم‌افزارها فقط چهار یا پنج قلم مناسب برای متن می‌توان یافت. قلم‌هایی که با فرم‌های تزیینی، دست خط و یا فرم‌های خاص و آزاد طراحی شده‌اند، هیچ‌یک برای کاربرد در یک متن مناسب نیستند. فرم خاص حروف قلم‌های ذکرشده علاوه بر آن‌که برای خوانده شدن متن را دشوار می‌کند، از وقار و تشخص صورت بیرونی متن می‌کاهد. بنابراین لازم است گزینش نوع قلم متن با توجه به محتوا با جدیت و دقت نظر بیشتری انجام گیرد.

\subsection{متن همراه با نمادها و فرمول‌های ریاضی}
مجموع اعضای $E$، آن هم وقتی که $E=\{0,\pm2,\pm4,\ldots\}$ برابر با صفر است. با دسته‌بندی، می‌توان به سادگی پاسخ را یافت. کافی است دسته‌بندی را به شکل $(0)+(-2+2)+(-4+4)+\cdots$ انجام داد که خاصیتِ قرینه‌بودنِ اعداد، بهتر به چشم می‌آید.
\begin{definition}[عدد خوب]
عدد طبیعیِ $k$، یک \emph{عدد خوب }است، اگر معادلۀ درجه دومِ
\[kx^2-(2k-1)x+\ln(k-0.1)=0\]
در بازۀ $(-\pi,\pi^2)$ جواب داشته‌باشد.
\end{definition}
در اینجا ممکن است پرسشی مطرح شود که آیا
\[\lim_{x\to\infty} f(2x-\alpha_0)=\cos^{-1}(0.02)
=\prod_{i=1}^{5000} \frac{A_\alpha B_{\frac{1}{2}-\alpha}}{2.06\times10^{-23}}
\]
در حالت کلی برای هر $\alpha\leq0$ برقرار است؟

\begin{proposition}[بخش‌پذیری بر $35$]
عددی بر $35$ بخش‌پذیر است که بر $7$ و $5$ بخش‌پذیر باشد.
\end{proposition}

\begin{proof}
چون $7$ و $5$ نسبت به یکدیگر اوّل هستند، پس اگر عددی بر آن‌ها بخش‌پذیر باشد، بر حاصل‌ضرب آن‌ها، یعنی $35$ نیز بخش‌پذیر است. 
\end{proof}

دربارۀ جمع توان دومِ اعداد طبیعی، در گزارۀ زیر آمده‌است.

\begin{proposition}
عدد $n\in\mathbb{N}$ دلخواه است. تساوی زیر برای این $n$ برقرار است
\[
1^2+2^2+3^2+\cdots+n^2=\sum_{k=1}^n k^2 = \frac{1}{6}(n)(n+1)(2n+1).
\]
\end{proposition}

\begin{proof}
اثبات با استقرای ریاضی. تمرین.
\end{proof}


\subsection{ویژگی‌های پیشرفته}
با فعال کردن هریک از گزینه‌های زیر، ویژگی‌های مربوط به آن نمایش داده می‌شوند.
\paragraph*{گزینۀ \lr{pnum}}
عددهااز حالتِ «عرضِ یکسان»، به «عرضِ متغیر» تبدیل می‌شوند.
\paragraph*{گزینۀ \lr{sefr}}
عدد صفر پارسی، از توخالی به توپر تبدیل می‌شود.

\paragraph*{گزینۀ \lr{shdw}}
تمام حروف، اعداد، نمادها، لاتین، پارسی و عربی، به حالتِ «سایه‌دار» تبدیل می‌شوند.


\subsection{حالت‌های فونتِ «پارسی متن»}

\subsubsection{عادی (\lr{Regular})}

پیش از آن‌که به تعریف حروف‌چینی بپردازیم، لازم است مقدمه‌ای دربارۀ خوشنویسی (خطاطی) و تفاوت آن با حروف‌چینی ارائه شود. شباهت و یکی دانستن حروف‌چینی و خوشنویسی بر دریافت و آشنایی ما از نوشتن با ماشین حروف‌چینی بی‌اثر نیست. خوشنویسی ظرف مدت کوتاهی پس از پیدایش خط، تولد یافته. خط دست‌نویس، مسیر غیرقابل پیش‌بینی و حسی را طی می‌کند تا به تکامل برسد. هنگام نوشتن با دست، خطاط از همه حالات نو و بدیع در گردش‌های قلم و نقطه‌گذاری‌ها، در ترکیب‌بندی نوشته استفاده می‌کند که شکل تکامل‌یافتۀ آن در ترکیبات و سیاه‌مشق‌های خوشنویسی کاملاً مشهود است.

\subsubsection{پررنگ (\lr{Bold})}
{\bfseries
پیش از آن‌که به تعریف حروف‌چینی بپردازیم، لازم است مقدمه‌ای دربارۀ خوشنویسی (خطاطی) و تفاوت آن با حروف‌چینی ارائه شود. شباهت و یکی دانستن حروف‌چینی و خوشنویسی بر دریافت و آشنایی ما از نوشتن با ماشین حروف‌چینی بی‌اثر نیست. خوشنویسی ظرف مدت کوتاهی پس از پیدایش خط، تولد یافته. خط دست‌نویس، مسیر غیرقابل پیش‌بینی و حسی را طی می‌کند تا به تکامل برسد. هنگام نوشتن با دست، خطاط از همه حالات نو و بدیع در گردش‌های قلم و نقطه‌گذاری‌ها، در ترکیب‌بندی نوشته استفاده می‌کند که شکل تکامل‌یافتۀ آن در ترکیبات و سیاه‌مشق‌های خوشنویسی کاملاً مشهود است.
}

\subsubsection{خوابیده به راست (\lr{Italic})}
{\itshape
پیش از آن‌که به تعریف حروف‌چینی بپردازیم، لازم است مقدمه‌ای دربارۀ خوشنویسی (خطاطی) و تفاوت آن با حروف‌چینی ارائه شود. شباهت و یکی دانستن حروف‌چینی و خوشنویسی بر دریافت و آشنایی ما از نوشتن با ماشین حروف‌چینی بی‌اثر نیست. خوشنویسی ظرف مدت کوتاهی پس از پیدایش خط، تولد یافته. خط دست‌نویس، مسیر غیرقابل پیش‌بینی و حسی را طی می‌کند تا به تکامل برسد. هنگام نوشتن با دست، خطاط از همه حالات نو و بدیع در گردش‌های قلم و نقطه‌گذاری‌ها، در ترکیب‌بندی نوشته استفاده می‌کند که شکل تکامل‌یافتۀ آن در ترکیبات و سیاه‌مشق‌های خوشنویسی کاملاً مشهود است.
}

\subsubsection{خوابیده به راست، پررنگ (\lr{Bold Italic})}
{\bfseries\itshape
پیش از آن‌که به تعریف حروف‌چینی بپردازیم، لازم است مقدمه‌ای دربارۀ خوشنویسی (خطاطی) و تفاوت آن با حروف‌چینی ارائه شود. شباهت و یکی دانستن حروف‌چینی و خوشنویسی بر دریافت و آشنایی ما از نوشتن با ماشین حروف‌چینی بی‌اثر نیست. خوشنویسی ظرف مدت کوتاهی پس از پیدایش خط، تولد یافته. خط دست‌نویس، مسیر غیرقابل پیش‌بینی و حسی را طی می‌کند تا به تکامل برسد. هنگام نوشتن با دست، خطاط از همه حالات نو و بدیع در گردش‌های قلم و نقطه‌گذاری‌ها، در ترکیب‌بندی نوشته استفاده می‌کند که شکل تکامل‌یافتۀ آن در ترکیبات و سیاه‌مشق‌های خوشنویسی کاملاً مشهود است.
}


\end{document}